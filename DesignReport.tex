%%%%%%%%%%%%%%%%%%%%%%%%%%%%%%%%%%%%%%%%%
% University/School Laboratory Report
% LaTeX Template
% Version 3.0 (4/2/13)
%
% This template has been downloaded from:
% http://www.LaTeXTemplates.com
%
% Original author:
% Linux and Unix Users Group at Virginia Tech Wiki 
% (https://vtluug.org/wiki/Example_LaTeX_chem_lab_report)
%
% License:
% CC BY-NC-SA 3.0 (http://creativecommons.org/licenses/by-nc-sa/3.0/)
%
%%%%%%%%%%%%%%%%%%%%%%%%%%%%%%%%%%%%%%%%%

%----------------------------------------------------------------------------------------
%	PACKAGES AND DOCUMENT CONFIGURATIONS
%----------------------------------------------------------------------------------------

\documentclass{article}

\usepackage{mhchem} % Package for chemical equation typesetting
\usepackage{siunitx} % Provides the \SI{}{} command for typesetting SI units
\usepackage{fullpage} % widens margins in the document

\usepackage{graphicx} % Required for the inclusion of images

\usepackage{fancyhdr}
\pagestyle{fancy}
\lhead{Linda Tseng - 304272081}
\rhead{Spencer Tung - 004355860}

\headheight = 10pt
\headsep = 15pt

\setlength\parindent{0pt} % Removes all indentation from paragraphs

\renewcommand{\labelenumi}{\alph{enumi}.} % Make numbering in the enumerate environment by letter rather than number (e.g. section 6)

%\usepackage{times} % Uncomment to use the Times New Roman font

%----------------------------------------------------------------------------------------
%	DOCUMENT INFORMATION
%----------------------------------------------------------------------------------------

\title{Defending Against a Process Overload Attack \\ CS 111} % Title

\author{Linda \textsc{Tseng}
		\and
		Spencer \textsc{Tung}} % Author name

\date{\today} % Date for the report

\begin{document}

\maketitle % Insert the title, author and date

\begin{center}
\begin{tabular}{l r} 
Partners: & Li Tseng 304272081 \\ % Partner names
& Spencer Tung 004355860 \\
Instructor: & Professor Reiher % Instructor/supervisor
\end{tabular}
\end{center}

% If you wish to include an abstract, uncomment the lines below
% \begin{abstract}
% Abstract text
% \end{abstract}

%----------------------------------------------------------------------------------------
%	SECTION 1
%----------------------------------------------------------------------------------------

\section{Security Objective}

Our goal is to prevent a process overload attack when using our shell. A 
process overload attack occurs when a process forks more processes than the
underlying operating system can handle. If the operating system is not prepared
to handle this kind of attack, it may cause the kernel to crash. This can 
occur either maliciously or unintentionally\footnote{This actually occurred
during our implementation of lab1c, in which we accidentally created an 
infinite loop while forking our parallel processes. It took another half hour
to regain control long enough to kill all of the rogue processes}, but in 
either case, we must be prepared to mitigate or kill the processes before it 
gets out of control. \\ 

Our implementation will seek to control the flow of the attack and ultimately
kill the processes before they cause our kernel to crash.


% To determine the atomic weight of magnesium via its reaction with oxygen and to study the stoichiometry of the reaction (as defined in \ref{definitions}):\\

% \begin{center}\ce{2 Mg + O2 -> 2 MgO}\end{center}

% % If you have more than one objective, uncomment the below:
% %\begin{description}
% %\item[First Objective] \hfill \\
% %Objective 1 text
% %\item[Second Objective] \hfill \\
% %Objective 2 text
% %\end{description}

% \subsection{Definitions}
% \label{definitions}
% \begin{description}
% \item[Stoichiometry]
% The relationship between the relative quantities of substances taking part in a reaction or forming a compound, typically a ratio of whole integers.
% \item[Atomic mass]
% The mass of an atom of a chemical element expressed in atomic mass units. It is approximately equivalent to the number of protons and neutrons in the atom (the mass number) or to the average number allowing for the relative abundances of different isotopes. 
% \end{description} 
 
%----------------------------------------------------------------------------------------
%	SECTION 2
%----------------------------------------------------------------------------------------

\section{Design Overview}

To start, we define a maximum number of processes that our shell (which is 
itself a process) can fork at any given time. This will be our \textbf{hard  
limit}; upon reaching this limit, we will kill the shell and any processes that
were forked from it to begin with. By checking the group process ID (GPID) of 
our shell function, we will be able to count how many other processes have the 
same GPID. We can also verify this through \textbf{pstree}, which creates a
tree of processes that we can refer to when forking. \\
We will also attempt to ensure fairness towards other running processes in the
system by lowering the priority of our shell if it starts to generate a large
number of forked processes. This will be our \textbf{soft limit}, as we will
attempt to slow the forking process first when we reach this limit, rather 
than kill it outright. We assume that when a process hits this limit, it is
possible that it is not a malicious program, and that it just needs more forked
processes than the norm. Therefore, we slow down the forking process to allow 
more time for its previously forked processes to catch up and finish. In the
event that it is indeed a malicious program (or a process that unintentionally 
created an infinite forking loop), then we should take appropriate action when 
we reach the hard limit and kill the process. \\

% \begin{tabular}{ll}
% Mass of empty crucible & \SI{7.28}{g}\\
% Mass of crucible and magnesium before heating & \SI{8.59}{g}\\
% Mass of crucible and magnesium oxide after heating & \SI{9.46}{g}\\
% Balance used & \#4\\
% Magnesium from sample bottle & \#1
% \end{tabular}

%----------------------------------------------------------------------------------------
%	SECTION 3
%----------------------------------------------------------------------------------------

\section{The Security Principles of Saltzer and Schroeder}
Blahblahblah \\

% \begin{tabular}{ll}
% Mass of magnesium metal & = \SI{8.59}{g} - \SI{7.28}{g}\\
% & = \SI{1.31}{g}\\
% Mass of magnesium oxide & = \SI{9.46}{g} - \SI{7.28}{g}\\
% & = \SI{2.18}{g}\\
% Mass of oxygen & = \SI{2.18}{g} - \SI{1.31}{}\\
% & = \SI{0.87}{g}
% \end{tabular}\\
% Because of this reaction, the required ratio is the atomic weight of magnesium: \SI{16.00}{g} of oxygen as experimental mass of Mg: experimental mass of oxygen or $\frac{x}{1.31}=\frac{16}{0.87}$ from which, $M_{\ce{Mg}} = 16.00 \times \frac{1.31}{0.87} = 24.1 = \SI{24}{g/mol}$ (to two significant figures).

%----------------------------------------------------------------------------------------
%	SECTION 4
%----------------------------------------------------------------------------------------

\section{Threat Model}

Blahblahblah \\
% The atomic weight of magnesium is concluded to be \SI{24}{g/mol}, as determined by the stoichiometry of its chemical combination with oxygen. This result is in agreement with the accepted value.

% \begin{figure}[h]
% \begin{center}
% \includegraphics[width=0.65\textwidth]{placeholder} % Include the image placeholder.png
% \caption{Figure caption.}
% \end{center}
% \end{figure}

%----------------------------------------------------------------------------------------
%	SECTION 5
%----------------------------------------------------------------------------------------

\section{Penetration Testing}
Monitor the total number of processes
In the first part of our approach, we want to monitor the total number of the processes on the shell, and kill it if it exceeds the given maximum number of processes.
We did this implementation in the function watchdog in execute-command.c.  To get the process forked from timetrash, we scan the processes in folder /proc and pass the information in the file to get the gid and pid from it.  Then, we can distinguish the processes with same gid as timetrash.  As the processes with same gid as timetrash increase, we update the number of process accordingly.  When it exceeds the given maximum number of processes, we will kill all the processes with same gid.  So that timetrash can not fork anymore processes, or it may occupy too much resources on the shell, which may let the shell crash. \\



Environment \\
At first, we run our program on UCLA CS lion server.  However, as we tried to test our program and let it fork as much as it can, it will let the server out of usage.  After that, we change our environment to the cs111 distribution on virtual machine.  Even if we crash the shell, we just need to reboot it then everything is good again.  We won’t affect other users like what we did on the server. \\ 

% The accepted value (periodic table) is \SI{24.3}{g/mol} \cite{Smith:2012qr}. The percentage discrepancy between the accepted value and the result obtained here is 1.3\%. Because only a single measurement was made, it is not possible to calculate an estimated standard deviation. \\

% The most obvious source of experimental uncertainty is the limited precision of the balance. Other potential sources of experimental uncertainty are: the reaction might not be complete; if not enough time was allowed for total oxidation, less than complete oxidation of the magnesium might have, in part, reacted with nitrogen in the air (incorrect reaction); the magnesium oxide might have absorbed water from the air, and thus weigh ``too much." Because the result obtained is close to the accepted value it is possible that some of these experimental uncertainties have fortuitously cancelled one another.

%----------------------------------------------------------------------------------------
%	SECTION 6
%----------------------------------------------------------------------------------------

\section{Robustness Analysis}

First, by limiting the maximum number of processes on the shell, we can prevent our shell from being overloaded since it will stop forking when it already has a certain number of processes.
Second, by increasing the nice factor when a process trying to fork too many child processes, we can let it fork slower.  Hence, it won’t occupy the shared resources on the shell, which may let other users cannot use the resources. \\ 

% \begin{enumerate}
% \begin{item}
% The \emph{atomic weight of an element} is the relative weight of one of its atoms compared to C-12 with a weight of 12.0000000$\ldots$, hydrogen with a weight of 1.008, to oxygen with a weight of 16.00. Atomic weight is also the average weight of all the atoms of that element as they occur in nature.
% \end{item}
% \begin{item}
% The \emph{units of atomic weight} are two-fold, with an identical numerical value. They are g/mole of atoms (or just g/mol) or amu/atom.
% \end{item}
% \begin{item}
% \emph{Percentage discrepancy} between an accepted (literature) value and an experimental value is $\frac{|\mathrm{experimental result} - \mathrm{accepted result}|}{\mathrm{accepted result}}$.
% \end{item}
% \end{enumerate}

%----------------------------------------------------------------------------------------
%	SECTION 7
%----------------------------------------------------------------------------------------

\section{Challenges}


%----------------------------------------------------------------------------------------
%	BIBLIOGRAPHY
%----------------------------------------------------------------------------------------

\bibliographystyle{unsrt}

\bibliography{sample}

%----------------------------------------------------------------------------------------


\end{document}