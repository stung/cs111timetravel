%%%%%%%%%%%%%%%%%%%%%%%%%%%%%%%%%%%%%%%%%
% University/School Laboratory Report
% LaTeX Template
% Version 3.0 (4/2/13)
%
% This template has been downloaded from:
% http://www.LaTeXTemplates.com
%
% Original author:
% Linux and Unix Users Group at Virginia Tech Wiki 
% (https://vtluug.org/wiki/Example_LaTeX_chem_lab_report)
%
% License:
% CC BY-NC-SA 3.0 (http://creativecommons.org/licenses/by-nc-sa/3.0/)
%
%%%%%%%%%%%%%%%%%%%%%%%%%%%%%%%%%%%%%%%%%

%----------------------------------------------------------------------------------------
%	PACKAGES AND DOCUMENT CONFIGURATIONS
%----------------------------------------------------------------------------------------

\documentclass{article}

\usepackage{mhchem} % Package for chemical equation typesetting
\usepackage{siunitx} % Provides the \SI{}{} command for typesetting SI units
\usepackage{fullpage} % widens margins in the document

\usepackage{graphicx} % Required for the inclusion of images

\usepackage[left=1in, right=1in]{geometry}

\usepackage{fancyhdr}
\pagestyle{fancy}
\lhead{Li Tseng - 304272081}
\rhead{Spencer Tung - 004355860}

\headheight = 10pt
\headsep = 15pt

\setlength\parindent{0pt} % Removes all indentation from paragraphs

\renewcommand{\labelenumi}{\alph{enumi}.} % Make numbering in the enumerate environment by letter rather than number (e.g. section 6)

%\usepackage{times} % Uncomment to use the Times New Roman font

%----------------------------------------------------------------------------------------
%	DOCUMENT INFORMATION
%----------------------------------------------------------------------------------------

\title{Defending Against a Process Overload Attack \\ CS 111} % Title

\author{Linda \textsc{Tseng}
		\and
		Spencer \textsc{Tung}} % Author name

\date{\today} % Date for the report

\begin{document}

\maketitle % Insert the title, author and date

\begin{center}
\begin{tabular}{l r} 
Partners: & Li Tseng 304272081 \\ % Partner names
& Spencer Tung 004355860 \\
Instructor: & Professor Reiher % Instructor/supervisor
\end{tabular}
\end{center}

% If you wish to include an abstract, uncomment the lines below
% \begin{abstract}
% Abstract text
% \end{abstract}

%----------------------------------------------------------------------------------------
%	SECTION 1
%----------------------------------------------------------------------------------------

\section{Security Objective}

Our goal is to prevent a process overload attack when using our shell. A 
process overload attack occurs when a process forks more processes than the
underlying operating system can handle. If the operating system is not prepared
to handle this kind of attack, it may cause the kernel to crash. This can 
occur either maliciously or unintentionally\footnote{This actually occurred
during our implementation of lab1c, in which we accidentally created an 
infinite loop while forking our parallel processes. It took another half hour
to regain control long enough to kill all of the rogue processes}, but in 
either case, we must be prepared to mitigate or kill the processes before it 
gets out of control. \\ 

Our implementation will seek to control the flow of the attack and ultimately
kill the processes before they cause our kernel to crash.


% To determine the atomic weight of magnesium via its reaction with oxygen and to study the stoichiometry of the reaction (as defined in \ref{definitions}):\\

% \begin{center}\ce{2 Mg + O2 -> 2 MgO}\end{center}

% % If you have more than one objective, uncomment the below:
% %\begin{description}
% %\item[First Objective] \hfill \\
% %Objective 1 text
% %\item[Second Objective] \hfill \\
% %Objective 2 text
% %\end{description}

% \subsection{Definitions}
% \label{definitions}
% \begin{description}
% \item[Stoichiometry]
% The relationship between the relative quantities of substances taking part in a reaction or forming a compound, typically a ratio of whole integers.
% \item[Atomic mass]
% The mass of an atom of a chemical element expressed in atomic mass units. It is approximately equivalent to the number of protons and neutrons in the atom (the mass number) or to the average number allowing for the relative abundances of different isotopes. 
% \end{description} 
 
%----------------------------------------------------------------------------------------
%	SECTION 2
%----------------------------------------------------------------------------------------

\section{Design Overview}

To start, we define a maximum number of processes that our shell (which is 
itself a process) can fork at any given time. This will be our hard upper 
limit; upon reaching this limit, we will kill the shell and any processes that
were forked from it to begin with. \\


First, we aim to limit the maximum number of processes that our shell can fork at any given time.  If a process hits the limit, we will kill it if it continues to attempt to fork.
Second, we want to ensure fairness in other programs by checking the number of children that a parent has forked using pstree.  When the number of processes exceed a certain limit, we will start increasing its nice factor to slowdown its forking.

% \begin{tabular}{ll}
% Mass of empty crucible & \SI{7.28}{g}\\
% Mass of crucible and magnesium before heating & \SI{8.59}{g}\\
% Mass of crucible and magnesium oxide after heating & \SI{9.46}{g}\\
% Balance used & \#4\\
% Magnesium from sample bottle & \#1
% \end{tabular}

%----------------------------------------------------------------------------------------
%	SECTION 3
%----------------------------------------------------------------------------------------

\section{The Security Principles of Saltzer and Schroeder}

% \begin{tabular}{ll}
% Mass of magnesium metal & = \SI{8.59}{g} - \SI{7.28}{g}\\
% & = \SI{1.31}{g}\\
% Mass of magnesium oxide & = \SI{9.46}{g} - \SI{7.28}{g}\\
% & = \SI{2.18}{g}\\
% Mass of oxygen & = \SI{2.18}{g} - \SI{1.31}{}\\
% & = \SI{0.87}{g}
% \end{tabular}\\
% Because of this reaction, the required ratio is the atomic weight of magnesium: \SI{16.00}{g} of oxygen as experimental mass of Mg: experimental mass of oxygen or $\frac{x}{1.31}=\frac{16}{0.87}$ from which, $M_{\ce{Mg}} = 16.00 \times \frac{1.31}{0.87} = 24.1 = \SI{24}{g/mol}$ (to two significant figures).

%----------------------------------------------------------------------------------------
%	SECTION 4
%----------------------------------------------------------------------------------------

\section{Threat Model}

% The atomic weight of magnesium is concluded to be \SI{24}{g/mol}, as determined by the stoichiometry of its chemical combination with oxygen. This result is in agreement with the accepted value.

% \begin{figure}[h]
% \begin{center}
% \includegraphics[width=0.65\textwidth]{placeholder} % Include the image placeholder.png
% \caption{Figure caption.}
% \end{center}
% \end{figure}

%----------------------------------------------------------------------------------------
%	SECTION 5
%----------------------------------------------------------------------------------------

\section{Implementation}
In this section, we will discuss the details of our design implementation. We
have broken it down into the following modules:
\subsection{Polling the children}
To make our design work, we needed to change our current \verb+execute_command+
implementation. In our original implementation of lab 1b, we used 
\verb+waitpid+ to block our parent process, only returning once the child had
finished running. While perfectly adequate for the lab, we realized that in 
order for us to continuously update the number of newly spawned processes, we
would need to 

Instead of blocking in the parent process, waiting for the
child to finish, we instead introduce a \verb+WNOHANG+ flag, which indicates
that we will continue to poll the child periodically to check on its status. \\
Changing our implementation from blocking to polling had no noticeable effect
on our runtime. However, if we had a long running child (such as a process that
hits an infinite loop), it is possible that this could affect our run-time 
performance. Namely, when a process is \textbf{blocked}, it indicates to the 
kernel 
\subsection{Watchdog Function}
We created a watchdog function that would run in the parent process, keeping
track of the total number of processes and modifying the relative priority of

\subsubsection{Counting processes}


In the first part of our approach, we want to monitor the total number of the processes on the shell, and kill it if it exceeds the given maximum number of processes.
We did this implementation in the function watchdog in execute-command.c.  To get the process forked from timetrash, we scan the processes in folder /proc and pass the information in the file to get the gid and pid from it.  Then, we can distinguish the processes with same gid as timetrash.  As the processes with same gid as timetrash increase, we update the number of process accordingly.  When it exceeds the given maximum number of processes, we will kill all the processes with same gid.  So that timetrash can not fork anymore processes, or it may occupy too much resources on the shell, which may let the shell crash. \\

\subsubsection{Playing nice}
\subsubsection{Hitting the kill switch}

\subsection{Environment}
At first, we run our program on UCLA CS lion server.  However, as we tried to test our program and let it fork as much as it can, it will let the server out of usage.  After that, we change our environment to the cs111 distribution on virtual machine.  Even if we crash the shell, we just need to reboot it then everything is good again.  We won’t affect other users like what we did on the server. \\ 

%----------------------------------------------------------------------------------------
%	SECTION 6
%----------------------------------------------------------------------------------------

\section{Penetration Testing}

In this section, we will discuss the tests that we ran. This section will
include the step-by-step instructions on how to run our demo against both good
and malicious inputs.

\subsection{Running the demo}
\subsection{Designing the malicious input}
\subsection{Time Testing}

% The accepted value (periodic table) is \SI{24.3}{g/mol} \cite{Smith:2012qr}. The percentage discrepancy between the accepted value and the result obtained here is 1.3\%. Because only a single measurement was made, it is not possible to calculate an estimated standard deviation. \\

% The most obvious source of experimental uncertainty is the limited precision of the balance. Other potential sources of experimental uncertainty are: the reaction might not be complete; if not enough time was allowed for total oxidation, less than complete oxidation of the magnesium might have, in part, reacted with nitrogen in the air (incorrect reaction); the magnesium oxide might have absorbed water from the air, and thus weigh ``too much." Because the result obtained is close to the accepted value it is possible that some of these experimental uncertainties have fortuitously cancelled one another.

%----------------------------------------------------------------------------------------
%	SECTION 7
%----------------------------------------------------------------------------------------

\section{Robustness Analysis}

In this section, we will analyze how effective our implementation was at
detecting and killing process overload attacks, while still allowing other processes to go about their merry way.


First, by limiting the maximum number of processes on the shell, we can prevent our shell from being overloaded since it will stop forking when it already has a certain number of processes.
Second, by increasing the nice factor when a process trying to fork too many child processes, we can let it fork slower.  Hence, it won’t occupy the shared resources on the shell, which may let other users cannot use the resources. \\ 


% \begin{enumerate}
% \begin{item}
% The \emph{atomic weight of an element} is the relative weight of one of its atoms compared to C-12 with a weight of 12.0000000$\ldots$, hydrogen with a weight of 1.008, to oxygen with a weight of 16.00. Atomic weight is also the average weight of all the atoms of that element as they occur in nature.
% \end{item}
% \begin{item}
% The \emph{units of atomic weight} are two-fold, with an identical numerical value. They are g/mole of atoms (or just g/mol) or amu/atom.
% \end{item}
% \begin{item}
% \emph{Percentage discrepancy} between an accepted (literature) value and an experimental value is $\frac{|\mathrm{experimental result} - \mathrm{accepted result}|}{\mathrm{accepted result}}$.
% \end{item}
% \end{enumerate}

%----------------------------------------------------------------------------------------
%	SECTION 8
%----------------------------------------------------------------------------------------


\section{Results}

In this section, we will discuss our results, document the known bugs in our 
code, and close with the challenges that we encountered along the way.

\subsection{Division of Work}
Spencer implemented the process counting and killing, while Li implemented the
process priority setting.

\subsection{Known Bugs}
When we run our infinite loop fork test, our shell very quickly detects and
shuts down our shell. However, if we were to manually set the number of forked 
processes per process, our shell begins to exhibit undefined behavior between 
$i = 5$ and $i = 15$ forks per process. Anything less than 5 will complete
before our watchdog detects it, and anything more than 15 will be detected by
our watchdog and dealt with accordingly. But when we fork $5 - 15$ times per
process, the shell begins to complain that it is unable to open the file, and
seems to crash on its own volition. Other times, it would crash the first 
command, yet still run the rest of the commands in our shell script. We were 
unable to determine the cause of this bug.\\
In addition, possibly owing to the fact that our hard limit is set to only
$100$ processes, there is no significant deviation from a runtime of abou
$20ms$, regardless of how many forked processes we introduce.

\subsection{Challenges}
We encountered a number of challenges along the way. Outside of the usual 
debugging process, here were the main challenges that beset our progress:
\subsubsection{Method to count the processes}
\subsubsection{Polling the child's forked processes}
In our original implementation of lab 1b, we would used \verb+waitpid+ to block
the parent process, and only return when after the child had completed. 
\subsubsection{Updating the nice level for all forked processes}


%----------------------------------------------------------------------------------------
%	BIBLIOGRAPHY
%----------------------------------------------------------------------------------------

\bibliographystyle{unsrt}

\bibliography{sample}

%----------------------------------------------------------------------------------------


\end{document}